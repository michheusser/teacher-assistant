% !TEX TS-program = pdflatex
% !TEX encoding = UTF-8 Unicode

% This is a simple template for a LaTeX document using the "article" class.
% See "book", "report", "letter" for other types of document.

\documentclass[11pt]{article} % use larger type; default would be 10pt

\usepackage[utf8]{inputenc} % set input encoding (not needed with XeLaTeX)
\usepackage{amsmath}
\usepackage{amsfonts}

%%% Examples of Article customizations
% These packages are optional, depending whether you want the features they provide.
% See the LaTeX Companion or other references for full information.

%%% PAGE DIMENSIONS
\usepackage{geometry} % to change the page dimensions
\geometry{a4paper} % or letterpaper (US) or a5paper or....
% \geometry{margin=2in} % for example, change the margins to 2 inches all round
% \geometry{landscape} % set up the page for landscape
%   read geometry.pdf for detailed page layout information

\usepackage{graphicx} % support the \includegraphics command and options

% \usepackage[parfill]{parskip} % Activate to begin paragraphs with an empty line rather than an indent
\usepackage{parskip}
%%% PACKAGES
\usepackage{booktabs} % for much better looking tables
\usepackage{array} % for better arrays (eg matrices) in maths
\usepackage{paralist} % very flexible & customisable lists (eg. enumerate/itemize, etc.)
\usepackage{verbatim} % adds environment for commenting out blocks of text & for better verbatim
\usepackage{subfig} % make it possible to include more than one captioned figure/table in a single float
% These packages are all incorporated in the memoir class to one degree or another...

%%% HEADERS & FOOTERS
\usepackage{fancyhdr} % This should be set AFTER setting up the page geometry
\pagestyle{fancy} % options: empty , plain , fancy
\renewcommand{\headrulewidth}{0pt} % customise the layout...
\lhead{}\chead{}\rhead{}
\lfoot{}\cfoot{\thepage}\rfoot{}

%%% SECTION TITLE APPEARANCE
\usepackage{sectsty}
\allsectionsfont{\sffamily\mdseries\upshape} % (See the fntguide.pdf for font help)
% (This matches ConTeXt defaults)

%%% ToC (table of contents) APPEARANCE
\usepackage[nottoc,notlof,notlot]{tocbibind} % Put the bibliography in the ToC
\usepackage[titles,subfigure]{tocloft} % Alter the style of the Table of Contents
\renewcommand{\cftsecfont}{\rmfamily\mdseries\upshape}
\renewcommand{\cftsecpagefont}{\rmfamily\mdseries\upshape} % No bold!

%%% END Article customizations

%%% The "real" document content comes below...

\title{Analysis I: Übung 9}
\author{Michel Heusser}
%\date{} % Activate to display a given date or no date (if empty),
         % otherwise the current date is printed 

\begin{document}
\maketitle

\section{Theorie}

\subsection{Parameterisierung einer Expliziten Kurve}


\underline{Satz:} Eine Kurve $\Gamma = \{(x,y) | y = f(x), x \in [a,b] \} = \{(r,\varphi) | r = g(\varphi), \varphi \in [c,d]\} $ die in explizite Form gegeben ist (in kartesische- oder Polarkordinaten) kann immer Parameterisiert werden indem man $x(t) = t$ setzt und folglich $y(t) = f(t)$ wird (bzw. $\varphi(t) = t$ und $r(t) = g(t)$) zur Form $\Gamma = \{(x(t)=t,y(t)=f(t)) |t \in [a,b] \} = \{(r(t)=g(t), \varphi(t) = t | t \in [c,d])\}$

\subsubsection{Beispiel}

Parameterisiere die Kurve:

$\Gamma = \{(x,y) | y = \ln(x), x\in (0,\infty)\}$

$\Gamma: t \rightarrow  \left(\!
    \begin{array}{c}
     x(t)   \\
     y(t) 
    \end{array}
  \!\right) = \left(\!
    \begin{array}{c}
     t   \\
     \ln(t) 
    \end{array}
  \!\right), t \in (0, \infty)$\\\\

\subsubsection{Beispiel}

Parameterisiere die Kurve:

$\Gamma = \{(r,\varphi) | r = 2R\sin(\varphi), \; \varphi \in [0,\pi)\}$

$\Gamma: t \rightarrow  \left(\!
    \begin{array}{c}
     r(t)   \\
     \varphi(t) 
    \end{array}
  \!\right) = \left(\!
    \begin{array}{c}
     2R\sin(t)   \\
     t 
    \end{array}
  \!\right), t \in [0, \pi)$\\\\

Manchmal wird auch: $\Gamma: \varphi \rightarrow  \left(\!
    \begin{array}{c}
     2R\sin(\varphi)   \\
     \varphi 
    \end{array}
  \!\right), \varphi \in [0, \pi)$\\\\

\subsection{Steigung einer Parameterisierten Kurve}


\underline{Satz}: Die Steigung $m(t)$ einer nach $t$ parameterisierten Kurve wird wie folgt berechnet:

\begin{center}
$m(t) = \frac{\dot y(t)}{\dot x(t)}$
\end{center}

\subsubsection{Beispiel}

Finde die Steigung der Kurve:
$\Gamma: t \rightarrow  \left(\!
    \begin{array}{c}
     x(t)   \\
     y(t) 
    \end{array}
  \!\right) = \left(\!
    \begin{array}{c}
     t   \\
     3\cos(t) 
    \end{array}
  \!\right), t \in [0, \pi)$\\ am Punkt $(\frac{\pi}{4},\frac{1}{\sqrt{2}})$

Man sucht zuerst $t_0$ so dass $x(t_0)=\frac{\pi}{4}$ (oder $y(t_0) = \frac{1}{\sqrt{2}}$, es geht um das gleiche $t_0$). Wir finden $t_0 = \frac{\pi}{4}$. 

Danach findet man $m(t_0) = \frac{\dot y(t_0)}{\dot x(t_0)}=\frac{-3\sin(t_0)}{1} = -3\frac{1}{\sqrt{2}}$

\subsection{Parameterdarstellung einer Tangente}

\underline{Satz:} Die Tangente $T$ einer Kurve $\Gamma = \{(x(t),y(t)) |t \in [a,b] \}$ am Punkt $(x_0 = x(t_0),y_0 = y(t_0)$ hat die folgende Form:

\begin{center}

$T: t \rightarrow  \left(\!
    \begin{array}{c}
     x_0  \\
     y_0
    \end{array}
  \!\right) + t\cdot \vec u, \quad t\in \mathbb{R}\qquad \vec u = \frac{1}{\sqrt{\dot x^2(t_0)+\dot y^2(t_0)}}  \left(\!
    \begin{array}{c}
     \dot x(t_0)  \\
     \dot y(t_0)
    \end{array}
  \!\right), $ 

\end{center}

\emph{Bemerkung:} Hier ist $\vec u$ der normierte (und schon behandelte) Tangetialvektor

\subsubsection{Beispiel}

Finden sie die Tangente $T$ im Punkt $(\frac{1}{2},\sqrt{2})$ (kart. Koordinaten)

$\Gamma: t \rightarrow  \left(\!
    \begin{array}{c}
     x(t)   \\
     y(t) 
    \end{array}
  \!\right) = \left(\!
    \begin{array}{c}
     1/t   \\
     \sqrt{t} 
    \end{array}
  \!\right), t \in (0, \infty)$

Man sucht zuerst den ent. $t_0$ zum gesuchten Punkt. $x(t_0) = \frac{1}{2} \Rightarrow \frac{1}{t_0} = \frac{1}{2} \Rightarrow t_0 = 2$. Danach findet man den normierten Tangentialvektor $\vec u$:\\\\

$\vec u = \frac{1}{\sqrt{\dot x^2(t_0)+\dot y^2(t_0)}}  \left(\!
    \begin{array}{c}
     \dot x(t_0)  \\
     \dot y(t_0)
    \end{array}
  \!\right)  = \frac{1}{\sqrt{ (-\frac{1}{t_0^2})^2+ (\frac{1}{2\sqrt{t_0}})^2 }}  \left(\!
    \begin{array}{c}
      -\frac{1}{t_0^2}  \\
      \frac{1}{2\sqrt{t_0}}
    \end{array}
  \!\right) = \frac{1}{\sqrt{ \frac{1}{8}+ \frac{1}{8}}}  \left(\!
    \begin{array}{c}
      -\frac{1}{4}  \\
      \frac{1}{2\sqrt{2}}
    \end{array}
  \!\right) =  \left(\!
    \begin{array}{c}
      -\frac{1}{2}  \\
      \frac{1}{\sqrt{2}}
    \end{array}
  \!\right), $

Es folgt dann:

$T: t \rightarrow  \left(\!
    \begin{array}{c}
     \frac{1}{2} \\
     \sqrt{2}
    \end{array}
  \!\right) + t \cdot \left(\!
    \begin{array}{c}
      -\frac{1}{2}  \\
      \frac{1}{\sqrt{2}}
    \end{array}
  \!\right), t \in \mathbb{R}$

Mann könnte auch die Kurve $\Gamma$ explizit darstellen und dann die bekannte Formel $t_{x_0}(x) = f'(x_0)(x-x_0) + f(x_0)$ aufstellen und parameterisieren (gleiche Resultat). Das Vorteilhafte an der parameterisierte Tangente, ist dass alle mögliche Richtungen möglich sind, inkl. Richtungen parallel zur $y$-Achse

\subsection{Bogenlenge einer Kurve}
\underline{Definition:} Die {\bf Bogenlänge} einer parameterisierten Kurve $\Gamma = \{(x(t),y(t)) |t \in [a,b] \}$ ist die Funktion $s(t)$, die die Länge der Kurve $\Gamma$ von $t=a$ bis jedem $t$ gibt. Sie wird wie folgt berechnet:

\begin{center}
$s(t) = \int \limits_a^t \sqrt{\dot x^2(t') + \dot y^2(t')} \; \text{d}t' $
\end{center}

\emph{Bemerkung:} Weil wir eben von $a$ bis $t$ integrieren, brauchen wir eine Laufvariable die von $a$ zu einem bestimmten "fixen" $t$ integriert. Die nennen wir $t'$.

\subsubsection{Beispiel:}

Berechne die Bogenlenge der Kurve:
$\Gamma: t \rightarrow  \left(\!
    \begin{array}{c}
     x(t)   \\
     y(t) 
    \end{array}
  \!\right) = \left(\!
    \begin{array}{c}
     1/t   \\
     \sqrt{t} 
    \end{array}
  \!\right), t \in (1, \infty)$\\

$s(t) =  \int \limits_1^t \sqrt{\dot x^2(t') + \dot y^2(t')} \; \text{d}t'  = \int \limits_1^t \sqrt{ (-\frac{1}{t'^2})^2+ (\frac{1}{2\sqrt{t'}})^2 } \; \text{d}t' = \int \limits_1^t \sqrt{ \frac{1}{t'^4}+ \frac{1}{4t'} } \; \text{d}t' = ...$\\\\
$\Rightarrow s(t) = ...$
 
\subsection{Parameterisierung nach der Bogenlänge}

\underline{Satz:} Eine parameterisierte Kurve $\Gamma = \{(x(t),y(t)) |t \in [a,b] \}$ kann {\bf nach ihrer Bogenlänge} parameterisiert werden, indem man zuerst $s(t)$ berechnet und dann man die Funktion invertiert, so dass man $t(s)$ bekommt. Dieses $t(s)$ kann man dann in die Parameterisierung nach $t$ einsetzen zu $\Gamma = \{ (x(t(s)),y(t(s)) \ s \in [s(a), s(b)]\} = \{ (x(s), y(s)) | s \in [c , d]\}$

\subsubsection{Beispiel}

Die Bogenlänge der Kurve $\Gamma: t \rightarrow  \left(\!
    \begin{array}{c}
     x(t)   \\
     y(t) 
    \end{array}
  \!\right) = \left(\!
    \begin{array}{c}
     t   \\
     t^\frac{3}{2}
    \end{array}
  \!\right), t \in (-\frac{4}{9}, \infty)$\\ ist:$s(t) = \frac{8}{27}(1+\frac{9}{4}t)^\frac{3}{2}$\\

Wir lösen jetzt nach $t$ nach: $ \Rightarrow t(s) = s^\frac{2}{3} - \frac{4}{9}$. Durch das Einsetzen in die Kurvenfunktion wir bekommen:\\

$\Gamma: s \rightarrow  \left(\!
    \begin{array}{c}
     x(t(s))   \\
     y(t(s)) 
    \end{array}
  \!\right) = \left(\!
    \begin{array}{c}
     t(s)   \\
     t(s)^\frac{3}{2}
    \end{array}
  \!\right) = \left(\!
    \begin{array}{c}
     s^\frac{2}{3} - \frac{4}{9}   \\
     (s^\frac{2}{3} - \frac{4}{9})^\frac{3}{2}
    \end{array}
  \!\right), s \in (0, \infty)$\\

$\Gamma$ ist jetzt nach der Bogenlänge parameterisiert!

\subsection{Krümmung und Krümmungsradius}

\underline{Definition:} Die Krümmung einer nach ihrer Bogenlänge parameterisierte Kurve $\Gamma = \{(x(s),y(s)) |s \in [c,d] \}$ ist definiert als die Steigungsänderung, wenn man sich entlang der Kurve bewegen würde (deshalb Bogenlänge). Oder anders gesagt: $k(s) := \frac{\text{d}\alpha(s)}{\text{d}s}$ mit $\alpha(s) = \arctan(\frac{\dot y(s)}{\dot x(s)})$. Eingesetzt bekommt man:
\begin{center}
$k(s) = \frac{\ddot y \dot x - \dot y \ddot x}{\dot x^2 +\dot y^2}$
\end{center}

\underline{Satz:} Wenn eine Kurve {\bf nicht nach der Bogenlänge} parameterisiert ist ($\Gamma =  \{(x(t),y(t)) |t \in [a,b] \}$). Dann berechnet man die Kurve nach der oberen Definition der Krümmung:

\begin{center}
$k(t) = \frac{\ddot y \dot x - \dot y \ddot x}{(\dot x^2 +\dot y^2)^\frac{3}{2}}$\\
\end{center}

\emph{Beweis}: Man berechnet die Bogenlänge $s(t) = f(t)$ und daraus $t(s) = f^{-1}(s)$. Damit schreibt man $k(s)$ als $k(t(s))$ und man benutzt die Definition von Krümmung und die Kettenregel $k(t(s)) := \frac{d\alpha}{ds} = \frac{d\alpha}{dt}\cdot \frac{dt}{ds}.$ Um $\frac{ds}{dt}$ zu berechnen benutzt man die Definition von Bogenlänge und den Hauptsatz der Integralrechnung ("Die Umkehrung von Integrieren ist Ableiten"):  $\frac{ds}{dt} = \frac{d}{dt}\int \limits_a^t \sqrt{\dot x^2(t') + \dot y^2(t')} \; \text{d}t' = \sqrt{\dot x^2(t) + \dot y^2(t)}$ 



\underline{Satz:} Der Krümmungsradius von einer parameterisierten Kurve $\Gamma$ zum Punkt $(x(t),y(t))$ ist der Radius des am besten angepassten Kreises am Punkt $(x(t),y(t)$ und es wird wie folgt berechnet:

\begin{center}
$R(t) = \frac{1}{k(t)} \qquad$ wobei $k(t)$ die Krümmung ist.
\end{center}

\emph{Bemerkung:} Der am besten angepasstsen Kreis am Punkt $(x_0,y_0)$ berührt Tangential die Kurve $\Gamma$. Der Verbindungsvektor zwischen $(x_0,y_0)$ und der Mittelpunkt des Kreises liegt {\bf senkrecht} auf die tangentiale Richtung der Kurve am gleichen Punkt.

\section{Tipps zur Übung}

\subsection{Online-Teil}
\begin{itemize}
\item Frage 1: Parameterisierung. Krümung aufstellen. Maximum finden.
\item Frage 2: Gleich wie beim Schnittpunkt von zwei Kurven  $y=f_1(x)$ und $y = f_2(x)$, man sucht den Punkt $(x,y)$ der beide Gleichungen erfüllt. Dadurch kriegt man zwei Gleichungen ($y=f_1(x)$ und $y = f_2(x)$) und zwei Unbekannten ($x$ und $y$) die man nach den Unbekannten lösen kann. Analog, sucht man den Punkt (in Polarkoordinaten) $(r,\phi)$ der in beiden Kurven ist, also beide Kurvegleichunen erfüllt. Man transformiert den Punkt in Kartesische Koordinate und findet die Tangente auf dem Punkt P (Theorie)
\item Frage 3: Kurve parameterisieren und in kart. Koordinaten umwandeln.  
\end{itemize}




\end{document}



