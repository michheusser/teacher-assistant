% !TEX TS-program = pdflatex
% !TEX encoding = UTF-8 Unicode

% This is a simple template for a LaTeX document using the "article" class.
% See "book", "report", "letter" for other types of document.

\documentclass[11pt]{article} % use larger type; default would be 10pt

\usepackage[utf8]{inputenc} % set input encoding (not needed with XeLaTeX)

\usepackage{amsfonts}

%%% Examples of Article customizations
% These packages are optional, depending whether you want the features they provide.
% See the LaTeX Companion or other references for full information.

%%% PAGE DIMENSIONS
\usepackage{geometry} % to change the page dimensions
\geometry{a4paper} % or letterpaper (US) or a5paper or....
% \geometry{margin=2in} % for example, change the margins to 2 inches all round
% \geometry{landscape} % set up the page for landscape
%   read geometry.pdf for detailed page layout information

\usepackage{graphicx} % support the \includegraphics command and options

% \usepackage[parfill]{parskip} % Activate to begin paragraphs with an empty line rather than an indent
\usepackage{parskip}
%%% PACKAGES
\usepackage{booktabs} % for much better looking tables
\usepackage{array} % for better arrays (eg matrices) in maths
\usepackage{paralist} % very flexible & customisable lists (eg. enumerate/itemize, etc.)
\usepackage{verbatim} % adds environment for commenting out blocks of text & for better verbatim
\usepackage{subfig} % make it possible to include more than one captioned figure/table in a single float
% These packages are all incorporated in the memoir class to one degree or another...

%%% HEADERS & FOOTERS
\usepackage{fancyhdr} % This should be set AFTER setting up the page geometry
\pagestyle{fancy} % options: empty , plain , fancy
\renewcommand{\headrulewidth}{0pt} % customise the layout...
\lhead{}\chead{}\rhead{}
\lfoot{}\cfoot{\thepage}\rfoot{}

%%% SECTION TITLE APPEARANCE
\usepackage{sectsty}
\allsectionsfont{\sffamily\mdseries\upshape} % (See the fntguide.pdf for font help)
% (This matches ConTeXt defaults)

%%% ToC (table of contents) APPEARANCE
\usepackage[nottoc,notlof,notlot]{tocbibind} % Put the bibliography in the ToC
\usepackage[titles,subfigure]{tocloft} % Alter the style of the Table of Contents
\renewcommand{\cftsecfont}{\rmfamily\mdseries\upshape}
\renewcommand{\cftsecpagefont}{\rmfamily\mdseries\upshape} % No bold!

%%% END Article customizations

%%% The "real" document content comes below...

\title{Analysis I: Übung 4}
\author{Michel Heusser}
%\date{} % Activate to display a given date or no date (if empty),
         % otherwise the current date is printed 

\begin{document}
\maketitle

\section{Theorie}

\subsection{Injektivität, Surjektivität, Bijektivität}
\underline{Definition}: Eine Funktion $f: X \rightarrow Y$, $x \rightarrow f(x)$ heisst {\bf surjektiv} falls:
\begin{itemize}  
\item Formal: $\forall y \in Y: \exists x \in X: f(x) = y$\\  $(\Rightarrow W(f) = Y)$
\item Formal wörtlich: Für jedes $y$ in $Y$ existiert (mindestens) ein $x$ in $X$, so dass $f(x) = y$
\item Intuitiv: Jedes Element vom Zielbereich ($Y$) wird von der Funktion "benutzt"... also der Wertebereich ist der ganze Zielbereich. 
\end{itemize}





\underline{Definition}: Eine Funktion $f: X \rightarrow Y$, $x \rightarrow f(x)$ heisst {\bf injektiv} falls:
\begin{itemize}  
\item Formal: $\forall x_1, x_2 \in X: (f(x_1) = f(x_2) \Rightarrow x_1 = x_2)$
\item Formal wörtlich: Der einzige Weg zwei stellen $x_1$ und $x_2$ (im Definitionsbereich von $f$) so zu wählen, dass man beide den gleichen Funktionswert bekommen ($f(x_1) = f(x_2)$) ist wenn beide genau die gleiche Stelle sind ($x_1=x_2$). Anders gesagt, es existiert keine zwei stellen mit dem gleichen Funktionswert. 
\item Intuitiv: Jeder $y$ hat einen "exklusiven" $x$.
\end{itemize}


\underline{Definition}: Eine Funktion $f: X \rightarrow Y$, $x \rightarrow f(x)$ heisst {\bf bijektiv} falls:
\begin{itemize}  
\item Formal wörtlich: $f$ ist surjektiv und bijektiv
\item Intuitiv: Jedes element im Zielbereich hat einen eindeutigen Element im Definitionsbereich. Oder anders gesagt, alle werte vom Zielbereich und Definitionsbereich werden benutzt und haben einen eindeutigen Partner.
\end{itemize}


\subsubsection{Beispiel}


$f: A \rightarrow B$, $x \rightarrow f(x)$ mit

$A= \{a,b,c\}$\\
$B = \{x,y\}$
\begin{table}[h!]
    \begin{tabular}{|l|l|}
        \hline
        $x$ & $f(x)$ \\ \hline
        $a$ & $x$    \\ 
        $b$ & $x$    \\ 
        $c$ & $y$    \\ 
        \hline
    \end{tabular}
\end{table}
\begin{itemize}
\item Surjektiv
\item Nicht injektiv
\end{itemize}

\subsubsection{Beispiele}


$f: A \rightarrow B$, $x \rightarrow f(x)$ mit

$A= \{a,b\}$\\
$B = \{x,y,z\}$
\begin{table}[h!]
    \begin{tabular}{|l|l|}
        \hline
        $x$ & $f(x)$ \\ \hline
        $a$ & $x$    \\ 
        $b$ & $y$    \\  
        \hline
    \end{tabular}
\end{table}
\begin{itemize}
\item Nicht Surjektiv
\item Injektiv
\end{itemize}

\subsubsection{Beispiele}

$f: A \rightarrow B$, $x \rightarrow f(x)$ mit

$A= [1,3]$\\
$B = [1,28]$\\

$f(x) = x^3$\\

\begin{itemize}
\item Nicht Surjektiv ($W(f) = [1,27] \neq [1,28] = B$)
\item Injektiv
\end{itemize}

$f: A \rightarrow B$, $x \rightarrow f(x)$ mit

$A= \mathbb{R}^+$\\
$B = (0,1]$\\

$f(x) = \frac{1}{1+(x-1)^2}$\\

\begin{itemize}
\item Surjektiv ($W(f) = B$)
\item Nicht injektiv (Graph! Keine "exklusivität")
\end{itemize}

$f: A \rightarrow B$, $x \rightarrow f(x)$ mit

$A=[0,3]$\\
$B = [0,3]$\\

$f(x) = \left \{ 
\begin{array}{l  l}
	\sin(x) & 0<x<\frac{\pi}{2}\\
	x & \frac{\pi}{2}<x<3\\
\end{array} \right.$ \\

\begin{itemize}
\item Nicht surjektiv ($W(f) \neq B$)
\item Injektiv (Graph!)
\end{itemize}

\subsection{Die Inverse Funktion}
\underline{Definition}: Die Funktion $f^{-1}: B \rightarrow A, x \rightarrow f^{-1}(x)$ heisst die {\bf inverse Funktion} von einer bijektiven funktion $f: A \rightarrow B, x \rightarrow f(x)$ falls gilt:
\begin{itemize}
\item Formal: $\forall x \in D(f) = A: f^{-1}(f(x))=x$ und $\forall x \in D(f^{-1}) = C: f(f^{-1}(x))=x$
\item Wörtlich Formal: Für jedes $x$ im Definitionsbereich von $f$ bzw. $f^{-1}$ bekomme ich das gleiche $x$ wenn ich zuerst $f$ und dann $f^{-1}$  bzw. $f^{-1}$ und dann $f$ anwende.
\item Intuitiv: Wenn eine Funktion mir den Partner $B$ im Wertebereich von einem Element $A$ im Definitionsbereich gibt, dann gibt mir seine Inverse den ursprünglichen Partner $A$ wenn ich ihr $B$ gebe.\\
\end{itemize}

\emph{Bemerkung:} Eine Inverse kann nur dann Definiert werden, wenn die Funktion $f: A \rightarrow B, x \rightarrow f(x)$ \underline{bijektiv} ist 
\begin{itemize}
\item Surjektivität: Man will dass der Definitionsbereich von $f$ der Zielbereich von $f^{-1}$ ist und umgekehrt (i.e. wenn $f: A \rightarrow B$, dann $f^{-1}:B \rightarrow A$), damit dass überhaupt möglich ist, müssen alle Werte von $f$ dann definiert sein für den umgekehrten Funktionsterm. Falls es Elemente im Zielbereich existieren würden, die nicht Funktionswerte von $f$ sind, wären diese dann nicht definiert für $f^{-1}$. Die "Umkehrung" der Mengen, wäre dann nicht möglich. Damit das nicht passiert, muss die Bildmenge von $f$ genau der Wertebereich sein ($B = W(f)$ und $A = W(f^{-1}$) und deswegen folgt $ B = W(f) = D(f^{-1})$ und $A = D(f)=W(f^{-1})$. $f$ muss also surjektiv sein.
\item Injektivität: Kann mit der Definition einer Funktion verstehen werden. Eine Funktion ordnet jedem $x \in D(f) = A$ einen \emph{einzigen} $y \in W(f) \subset B$ zu. Obwohl jedes $x$ einen eindeutigen $y$ hat, kann es im Allgemeinen sein, dass zwei Stellen $x_1$ und $x_2$ (natürlich $x_1 \neq x_2$) den gleichen Funktionswert haben ($f(x_1)=f(x_2)$). Wenn das tatsächlich der Fall wäre, dann würde die inverse Funktion $f^{-1}$ bei einer Stelle zwei mögliche Funktionswerte haben, was die Definition einer Funktion widerspricht!\\\\
\end{itemize} 

\subsubsection{Beispiel}

$f: A \rightarrow B$, $x \rightarrow f(x)$\\
A = \{a,b,c,d\}
B= \{w,x,y,z\}

\begin{table}[h!]
    \begin{tabular}{|l|l|}
        \hline
        $x$ & $f(x)$ \\ \hline
        $a$ & $w$    \\ 
        $b$ & $x$    \\ 
        $c$ & $y$    \\ 
        $d$ & $z$    \\
        \hline
    \end{tabular}
\end{table}

Ist bijektiv. Man kann dann eine inverse definieren:

$f^{-1}: B \rightarrow A$, $x \rightarrow f^{-1}(x)$\\
A = \{a,b,c,d\}
B= \{w,x,y,z\}


\begin{table}[h!]
    \begin{tabular}{|l|l|}
        \hline
        $x$ & $f^{-1}(x)$ \\ \hline
        $w$ & $a$    \\ 
        $x$ & $b$    \\ 
        $y$ & $c$    \\ 
        $z$ & $d$    \\
        \hline
    \end{tabular}
\end{table}

\subsubsection{Beispiel}

$f: A \rightarrow B$, $x \rightarrow f(x)$\\
A = \{a,b,c,d\}
B= \{w,x,y\}

\begin{table}[h!]
    \begin{tabular}{|l|l|}
        \hline
        $x$ & $f(x)$ \\ \hline
        $a$ & $w$    \\ 
        $b$ & $x$    \\ 
        $c$ & $y$    \\ 
        $d$ & $y$    \\
        \hline
    \end{tabular}
\end{table}

Ist surjektiv aber nicht injektiv. Also keine Inverse kann definiert werden.

\subsubsection{Beispiel}

$f: A \rightarrow B$, $x \rightarrow f(x)$\\
A = \{a,b,c\}
B= \{w,x,y\}

\begin{table}[h!]
    \begin{tabular}{|l|l|}
        \hline
        $x$ & $f(x)$ \\ \hline
        $a$ & $x$    \\ 
        $b$ & $x$    \\ 
        $c$ & $z$    \\
        \hline
    \end{tabular}
\end{table}

Nicht surjektiv und nicht injektiv. Man kann keine Inverse definieren (Wie würde es sonst gehen?)

\subsubsection{Beispiel}
$f: A \rightarrow B$, $x \rightarrow f(x)$\\
A = \{a,b,c\}
B= \{w,x,y,z\}

\begin{table}[h!]
    \begin{tabular}{|l|l|}
        \hline
        $x$ & $f(x)$ \\ \hline
        $a$ & $w$    \\ 
        $b$ & $x$    \\ 
        $c$ & $y$    \\ 
        
        \hline
    \end{tabular}
\end{table}

Injektiv aber nicht surjektiv. Man könnte die umgekehrte Vorschrift eindeutig definieren (Tabelle umkehren), der 
Definitionsbereich der Umkehrfunktion müsste dann $D(f^{-1}) = \{w,x,y,z\}$ sein, aber für $z$ gäbe es kein Funktionswert ($f^{-1}{z} = ?$) also kann man eine Umkehrfunktion nicht definieren.

\subsection{Asymtoten}

\underline{Definition:} Die Funktion $g: A \rightarrow \mathbb{R}$, $x \rightarrow g(x)$ heisst {\bf Asymtote} von $f: (c, \infty) \rightarrow \mathbb{R}$, $x \rightarrow f(x)$ falls gilt:
\begin{itemize}
\item Formal: $\lim_{x \rightarrow \infty} (f(x) - g(x)) = 0$
\item Intuitiv: Beide Funktionen sich immer mehr annähern (und im Unendlichen berühren)
\end{itemize}

\emph{Bemerkung:} Die Frage hier ist: wie findet man Asymtoten zu einer Funkion $f$? Durch Polynomdivision kann man die Funktionen so Umformen, dass man anschaulich ein "Teil" der Funktion sehen kann, der im Unendlichen verschwindet, auch wenn die ursprüngliche Funktion $f$ nicht konvergiert.

\subsubsection{Beispiel}

$f: \mathbb{R} \setminus \{1\} \rightarrow \mathbb{R}$, $x \rightarrow f(x) = \frac{2x + 1}{x-1}$\\

Da $\frac{2x + 1}{x-1} = \frac{2x -2 + 3}{x-1} = 2+\frac{3}{x-1}$ (Genau, was man in der Polynomdivision macht), definieren wir $g: \mathbb{R}\rightarrow \mathbb{R}$,  $x \rightarrow g(x) := 2$. \\\\
Man kann dann leicht verifizieren, dass $g$ eine Asymptote von $f$ ist mit: $\lim_{x \rightarrow \infty} (f(x) - g(x)) = \lim_{x \rightarrow \infty} (2+ \frac{3}{x-1} - 2) = \lim_{x \rightarrow \infty} \frac{3}{x-1} = 0$

\subsubsection{Beispiel}
$f: \mathbb{R} \setminus \{3\} \rightarrow \mathbb{R}$, $x \rightarrow f(x) = \frac{x^2 + 2x -8}{x-3}$\\

Da $\frac{x^2 + 2x -8}{x-3}= x + 5 +\frac{7}{x-3}$, definieren wir $g: \mathbb{R}\rightarrow \mathbb{R}$,  $x \rightarrow g(x): = x+5$. \\\\
Wir sehen, dass $g$ eine Asymptote von $f$ ist: $\lim_{x \rightarrow \infty} (f(x) - g(x)) = \lim_{x \rightarrow \infty} (x+ 5\frac{7}{x-3} - (x+5)) = \lim_{x \rightarrow \infty} \frac{7}{x-3} = 0$ (\emph{Bemerkung:} $g$ ist eine Asymptote trotz Divergenz, $g$ könnte auch z.B. periodisch sein, solange die Definition der Asymptote erfüllt ist!)

\section{Tipps}
\subsection{Online-Teil}
\begin{itemize}
\item Frage 1: Definition von Monotonie (Handout: Übung 2). Graph Anschauen!
\item Frage 2: Definition von Monotonie. Bei $f(x) = x^r$ und $r \in \mathbb{R}$, wie ändert $r$ die Kurve? 
\item Frage 3: Definition von Injektiv. 
\item Frage 4: Definition einer Funktion und injektivität. $f: A \rightarrow B, x\rightarrow f(x)$, hier ist $A$ eine Menge von Zahlen und $B$ eine Menge von Trippel $(\cdot, \cdot, \cdot)$ (Anschaulich: Punkte im 3D-Raum). Injektivität heisst, jeder Element von $A$ hat einen "exklusiven" Element in $B$. Wann ist das erfüllt?
\item Frage 5: Zur Berechnung einer inversen Funktion $f^{-1}$ (falls die existiert!) Löst man die Gleichung $y = f(x)$ nach $x$ auf, und vertauscht man $y$ und $x$, man setzt dann das neue (vertauschte) $y = f^{-1}(x)$
\item Frage 6: Voraussetzung einer Umkehrbare funktion: Bijektivität. Wann ist sie erfüllt?
\item Frage 7: $W(\sin(x)) \rightarrow W(\sin^2(x)) \rightarrow W(\sin^2(x) + 1)$

\subsection{Aufgabe 2}
Damit man den Funktionsterm umkehren kann, muss man die substitution $z := 2^x \Rightarrow x = \frac{\log z}{\log 2}$ benutzen und dann nach $z$ auflösen. Man bekommt zwei lösungen, aber $z = 2^x > 0$ ist nur eine relevant. Was ist $W(f), W(f^{-1}), D(f), D(f^{-1})$? Ist die Surjektivität erfüllt? (Wenn $W(f) = D(f^{-1})$ und $D(f)=W(f^{-1})$ gilt, dann ist $f$ surjektiv) 
\subsection{Aufgabe 3}
\begin{itemize}
\item a) Der Bruch konvergiert gegen $c$ ($c$ noch unbekannt). Deshalb ist g(x):=c eine Asymptote.
\item b) Polynomdivision oder geeignete ergänzung 
\end{itemize}
\subsection{Aufgabe 4}
Definition von injektivität, surjektivität, bijektivität. Kreativ sein!
\end{itemize}

\end{document}

