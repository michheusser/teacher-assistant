% !TEX TS-program = pdflatex
% !TEX encoding = UTF-8 Unicode

% This is a simple template for a LaTeX document using the "article" class.
% See "book", "report", "letter" for other types of document.

\documentclass[11pt]{article} % use larger type; default would be 10pt

\usepackage[utf8]{inputenc} % set input encoding (not needed with XeLaTeX)

\usepackage{amsfonts}

%%% Examples of Article customizations
% These packages are optional, depending whether you want the features they provide.
% See the LaTeX Companion or other references for full information.

%%% PAGE DIMENSIONS
\usepackage{geometry} % to change the page dimensions
\geometry{a4paper} % or letterpaper (US) or a5paper or....
% \geometry{margin=2in} % for example, change the margins to 2 inches all round
% \geometry{landscape} % set up the page for landscape
%   read geometry.pdf for detailed page layout information

\usepackage{graphicx} % support the \includegraphics command and options

% \usepackage[parfill]{parskip} % Activate to begin paragraphs with an empty line rather than an indent
\usepackage{parskip}
%%% PACKAGES
\usepackage{booktabs} % for much better looking tables
\usepackage{array} % for better arrays (eg matrices) in maths
\usepackage{paralist} % very flexible & customisable lists (eg. enumerate/itemize, etc.)
\usepackage{verbatim} % adds environment for commenting out blocks of text & for better verbatim
\usepackage{subfig} % make it possible to include more than one captioned figure/table in a single float
% These packages are all incorporated in the memoir class to one degree or another...

%%% HEADERS & FOOTERS
\usepackage{fancyhdr} % This should be set AFTER setting up the page geometry
\pagestyle{fancy} % options: empty , plain , fancy
\renewcommand{\headrulewidth}{0pt} % customise the layout...
\lhead{}\chead{}\rhead{}
\lfoot{}\cfoot{\thepage}\rfoot{}

%%% SECTION TITLE APPEARANCE
\usepackage{sectsty}
\allsectionsfont{\sffamily\mdseries\upshape} % (See the fntguide.pdf for font help)
% (This matches ConTeXt defaults)

%%% ToC (table of contents) APPEARANCE
\usepackage[nottoc,notlof,notlot]{tocbibind} % Put the bibliography in the ToC
\usepackage[titles,subfigure]{tocloft} % Alter the style of the Table of Contents
\renewcommand{\cftsecfont}{\rmfamily\mdseries\upshape}
\renewcommand{\cftsecpagefont}{\rmfamily\mdseries\upshape} % No bold!

%%% END Article customizations

%%% The "real" document content comes below...

\title{Analysis I: Übung 5}
\author{Michel Heusser}
%\date{} % Activate to display a given date or no date (if empty),
         % otherwise the current date is printed 

\begin{document}
\maketitle

\section{Theorie}

\subsection{Differenzierbarkeit}


\underline{Definition}: Eine Funktion $f: X \rightarrow Y$, $x \rightarrow f(x)$ heisst {\bf differenzierbar in $x_0 \in D(f)$} falls:
\begin{itemize}  
\item Formal: $\exists c = \lim_{d \rightarrow 0} \frac{f(x_0+d)-f(x_0)}{d}$ mit $c\in \mathbb{R}$
\item Formal wörtlich: Der sogenannte \underline{Differentialquotient} $\frac{f(x_0+d)-f(x)}{d}$ zur Funktion $f$ hat den gleichen eindeutigen endlichen (reelen) Grenzwert $c$ wenn man $d$ gegen null von links \underline{und} rechts streben lässt.
\end{itemize}


\emph{Erinnerung:} Wenn man $\lim_{d \rightarrow 0} h(d) $ schreibt, impliziert man dass  $\lim_{d \rightarrow 0} h(d) = \lim_{d \rightarrow 0^+} h(d) = \lim_{d \rightarrow 0^-} h(d)$\\

\subsubsection{Beispiel}

$f: [0,2] \rightarrow \mathbb{R}$, $x \rightarrow f(x) = x^2 + 2x $

Ist $f(x)$ differenzierbar in $x_0 =1$?\\

$ \frac{f(x_0+d)-f(x_0)}{d} = \frac{f(1+d)-f(1)}{d} = \frac{((1+d)^2+2(1+d))-(1^2 + 2\cdot 1)}{d}=\frac{2d+d^2+2d}{d}=  4+d $\\\\
$ \lim_{d \rightarrow 0^-} \frac{f(x_0+d)-f(x_0)}{d} =  \lim_{d \rightarrow 0^-} 4+d = 4 $\\
$ \lim_{d \rightarrow 0^+} \frac{f(x_0+d)-f(x_0)}{d} =  \lim_{d \rightarrow 0^+} 4+d = 4 $\\
$\Rightarrow \lim_{d \rightarrow 0} \frac{f(x_0+d)-f(x_0)}{d} =4 $ (Es konvergiert!)

Die Funktion $f$ ist in $x_0 = 1$ differenzierbar\\\\

\subsubsection{Beispiel}

$f: \mathbb{R} \rightarrow \mathbb{R}$, $x \rightarrow f(x)$

$f(x) = \left \{ 
\begin{array}{l  l}
	\sin(x) & x>0\\
	-x & x\leq 0\\
\end{array} \right.$ \\

Ist $f$ in $x_0 = 0$ differenzierbar?\\

$ \frac{f(0+d)-f(0)}{d} = \left \{ 
\begin{array}{l  l}
	\frac{\sin(0+d)-\sin(0)}{d} = \frac{\sin{d}}{d} & d>0\\
	\frac{-(0+d)-(-(0))}{d} = -1 & d\leq 0\\
\end{array} \right.$ \\

$\lim_{d \rightarrow 0^-} \frac{f(0+d)-f(0)}{d} = -1$\\
$\lim_{d \rightarrow 0^+} \frac{f(0+d)-f(0)}{d} = \lim_{d \rightarrow 0^+} \frac{\sin(d)}{d} =1$\\

Die Funktion $f$ ist nicht in $x_0=0$ differenzierbar

\subsubsection{Beispiel}

$f: \mathbb{R} \rightarrow \mathbb{R}$, $x \rightarrow f(x)$

$f(x) = \left \{ 
\begin{array}{l  l}
	\sin(x) & x>0\\
	x & x\leq 0\\
\end{array} \right.$ \\

Ist $f$ in $x_0 = 0$ differenzierbar?\\

$ \frac{f(0+d)-f(0)}{d} = \left \{ 
\begin{array}{l  l}
	\frac{\sin(0+d)-\sin(0)}{d} = \frac{\sin{d}}{d} & d>0\\
	\frac{(0+d)-(0)}{d} = 1 & d\leq 0\\
\end{array} \right.$ \\

$\lim_{d \rightarrow 0^-} \frac{f(0+d)-f(0)}{d} = 1$\\
$\lim_{d \rightarrow 0^+} \frac{f(0+d)-f(0)}{d} = \lim_{d \rightarrow 0^+} \frac{\sin(d)}{d} =1$\\

Die Funktion $f$ ist in $x_0=0$ differenzierbar\\\\


\underline{Definition}: Eine Funktion $f: X \rightarrow Y$, $x \rightarrow f(x)$ heisst {\bf differenzierbar} falls:
\begin{itemize}  
\item Formal: $\forall x_0 \in D(f): \exists c = \lim_{d \rightarrow 0} \frac{f(x_0+d)-f(x_0)}{d}$ mit $c\in \mathbb{R}$
\item Formal wörtlich: Der Term $\frac{f(x_0+d)-f(x_0)}{d}$ hat bei jedem $x$ den gleichen eindeutigen endlichen (reelen) Grenzwert $c$ wenn man $d$ gegen null von links \underline{und} rechts streben lässt.\\\\
\end{itemize}

\subsubsection{Beispiel}

$f: [0,2] \rightarrow \mathbb{R}$, $x \rightarrow f(x) = x^2 + 2x $

Ist $f$ differenzierbar? (Also ist $f$ differenzierbar in alle $x_0 \in [0,2]?)$\\

$ \frac{f(x_0+d)-f(x_0)}{d} = \frac{((x_0+d)^2+2(x_0+d))-(x_0^2 + 2x_0)}{d}=\frac{2x_0d+d^2+2d}{d}= 2x_0 +d + 2 $\\\\
$ \lim_{d \rightarrow 0^-} \frac{f(x_0+d)-f(x_0)}{d} =  \lim_{d \rightarrow 0^-} 2x_0 + d + 2 = 2x_0 + 2 $\\
$ \lim_{d \rightarrow 0^+} \frac{f(x_0+d)-f(x_0)}{d} =  \lim_{d \rightarrow 0^+} 2x_0 + d + 2 = 2x_0 + 2 $\\
$\Rightarrow \lim_{d \rightarrow 0} \frac{f(x_0+d)-f(x_0)}{d} =2x_0 + 2 $ (Es konvergiert für jeden $x_0$!)

Die Funktion $f$ ist differenzierbar

\subsubsection{Beispiel}


$f: \mathbb{R} \rightarrow \mathbb{R}$, $x \rightarrow f(x) = 1 + \sin(x) $

Ist $f$ differenzierbar? \\

$ \frac{f(x_0+d)-f(x_0)}{d} = \frac{(\sin(x_0+d)+1)-(sin(x_0)+1)}{d} = \frac{(\sin(x_0)\cos(d)+\sin(d)\cos(x_0) + 1) - (\sin(x_0)+1)}{d} \\=\frac{(\sin(x_0)\cos(d)+\sin(d)\cos(x_0)) - \sin(x_0)}{d} = \frac{\sin{x_0}(\cos(d)-1)+\sin(d)\cos(x_0)}{d}\\ =\sin(x_0)\cdot\frac{(\cos(d)-1)}{d}+\cos(x_0)\cdot\frac{\sin(d)}{d}$\\\\

$ \lim_{d \rightarrow 0^-} \frac{f(x_0+d)-f(x_0)}{d} =  \lim_{d \rightarrow 0^-} (\sin(x_0)\frac{(\cos(d)-1)}{d}+\cos(x_0)\frac{\sin(d)}{d})\\= \sin(x_0)\cdot\lim_{d \rightarrow 0^-} \frac{(\cos(d)-1)}{d}+\cos(x_0)\cdot\lim_{d \rightarrow 0^-} \frac{\sin(d)}{d} = \sin(x_0) \cdot 0 +\cos(x_0)\cdot1\\ = \cos(x_0) $\\


$ \lim_{d \rightarrow 0^+} \frac{f(x_0+d)-f(x_0)}{d} =$ ...Analog...$= \sin(x_0) \cdot 0 +\cos(x_0)\cdot1 \\ =\cos(x_0)$\\


$\Rightarrow \lim_{d \rightarrow 0} \frac{f(x_0+d)-f(x_0)}{d} =\cos(x_0) $ (Es konvergiert für jeden $x_0$!)

Die Funktion $f$ ist differenzierbar

\subsection{Ableitung}

\underline{Definition}: Die Funktion $f': X \rightarrow Y$, $x \rightarrow f'(x)$ heisst {\bf Ableitung} der differenzierbaren Funktion $f: X \rightarrow Y$, $x \rightarrow f(x)$ und ist wie folgt definiert:
\begin{center}
$f'(x) = \frac{\mathrm{d}f}{\mathrm{d}x} :=  \lim_{d\rightarrow 0} \frac{f(x+d)-f(x)}{d}$\\
\end{center}


\emph{Bemerkung}:  Die Ableitung ist der Grenzwert zu jedem $x$ vom Term $\frac{f(x+d)-f(x)}{d}$ wenn man $d$ gegen $0$ streben lässt ($x$ ist hier dann nicht die "Variable" die sich bewegt!). Damit man eine Funktion von Grenzwerten überhaupt definieren kann, müssen die Grenzwerte existieren (also muss der erwähnte Term bei jedem $x$ immer konvergieren und zwar gegen den gleichen Grenzwert von beiden Seiten!), deswegen definiert man genau den Begriff von Differenzierbarkeit!\\


Seien $f: A \rightarrow B$, $x \rightarrow f(x)$ und $g: A \rightarrow C$, $x \rightarrow f(x)$  und in einem bestimmten $x$ differenzierbare Funktionen, dann gilt:
\begin{itemize}
\item $(f(x) + g(x))' = f'(x) + g'(x)$
\item $(C\cdot f(x))' = C\cdot f'(x)$
\item $(f(x)\cdot g(x))' = f'(x)\cdot g(x) + f(x) \cdot g'(x)$
\item $(\frac{(f(x)}{g(x)})' = \frac{f'(x)g(x)-f(x)g'(x)}{g(x)^2}$\\

\end{itemize}

Sei $f: A \rightarrow B$, $x \rightarrow f(x)$ eine invertierbare Funktion.
\begin{itemize}
\item $(f^{-1}(x))'=\frac{1}{f'(f^{-1}(x))}$\\
\end{itemize}

Sei $g: A \rightarrow B$, $x \rightarrow g(x)$ eine in einem bestimmten $x$ differenzierbare Funktion und $f: C \rightarrow D$, $x \rightarrow f(x)$ eine in jedem $g(x)$ differenzierbare Funktion dann gilt die {\bf Kettenregel}:
\begin{itemize}
\item $(f(g(x))'=f'(g(x))\cdot g'(x)$\\
\end{itemize}


\subsection{Tangente (Linearisierung)}

\underline{Definition}: Die (lineare) Funktion $t_{x_0}(x) = a\cdot x + b$ ist eine Tangente zu einer Funktion $f(x)$ an der stelle  $x_0$ falls gilt:

\begin{itemize}
\item $t'_{x_0}(x=x_0)=f'(x=x_0)$
\item $t_{x_0}(x=x_0) = f(x=x_0)$
\end{itemize}
$\Rightarrow$ $t_{x_0}(x) = f'(x_0)\cdot (x-x_0) + f(x_0)$\\

\emph{Bemerkung:} $t_{x_0}(x)$ ist die Linearisierung (oder Approximation 1. Ordnung) von $f$ an der Stelle $x_0$, weil es ähnliche werte hat als $f(x)$ in der Umgebung von $x_0$.


\subsection{Bekannte Ableitungen}

$f: A \rightarrow B$, $x \rightarrow f(x)$ und $f'$ ist ihre Ableitung:

\begin{table}[h]
\centering
    \begin{tabular}{|l|l|}
        \hline
        $f(x)$                       & $f'(x)$                   \\ \hline
        $c$                          & $0$                       \\ 
        $x^r$ mit $r \in \mathbb{R}$ & $r\cdot x^{r-1}$          \\ 
        $e^x$                        & $e^x$                     \\ 
        $\ln(x)$                     & $\frac{1}{x}$             \\ 
        $\sin(x)$                    & $\cos(x)$                 \\ 
        $\cos(x)$                    & $-\sin(x)$                \\ 
        $\tan(x)$                    & $1+\tan^2(x)$             \\ 
        $\arcsin(x)$                 & $\frac{1}{\sqrt{1-x^2}}$  \\ 
        $\arccos(x)$                 & $-\frac{1}{\sqrt{1-x^2}}$ \\ 
        $\arctan(x)$                 & $\frac{1}{1+x^2}$         \\ 
        $\sinh(x)$                   & $\cosh(x)$                \\ 
        $\cosh(x)$                   & $\sinh(x)$                \\
        \hline 
    \end{tabular}
\end{table}



\section{Tipps zur Übung}

\subsection{Online-Teil}
\begin{itemize}
\item Frage 1: Definition von Asymptote benutzen, falls erfüllt $\rightarrow$ Asymptote
\item Frage 2: Wie Frage 1
\item Frage 3: Definition: von Tangente benutzen
\item Frage 4: Nullstellen der Ableitung von $f$ sind stellen wo $f$ die Steigung gleich Null ist. Lokale Minimale- und Maximalstellen haben keine Steigung gleich Null. 
\item Frage 5: Definition einer Tangente: Parameter $a$ und $b$ mit der Bedingung einer Tangente ermitteln
\item Frage 6: Definition von $a^x$? Wie kann man $x^x$ dann anders Schreiben (eigentlich seine Definition)? Kettenregel benutzen!
\item Frage 7: Rechenregeln von Ableitungen benutzen!
\item Frage 8: Kettenregel und Ableitung von $e^x$.
\end{itemize}

\subsection{Aufgabe 2}
Rechenregeln und bekannte Ableitungen benutzen.

\subsection{Aufgabe 3}
$f(x) = \ln(x)$
Tangente $t_{x_0}(x)$ zu einem beliebigen festen $x_0$ berechnen ($x_0$ als Variable lassen). Danach berechnet man die Seiten des Dreieckes:
\begin{itemize}
\item Seite 1: Von der Höhe von Schnitt von $t_{x_0}{x}$ mit der $y$-Achse bis $f(x=x_0)$
\item Seite 2: Von $0$ bis $x_0$
\end{itemize}

\subsection{Aufgabe 4}
Sich fragen: Was will man Beweisen? Was ist die Definition von "ungerade"? Irgendmal $x$ durch $f(f^{-1}(x))$ ersetzen...

\end{document}

%$f: \mathbb{R} \rightarrow \mathbb{R}$, $x \rightarrow f(x)$
%
%$f(x) = \left \{ 
%\begin{array}{l  l}
%	\sin(x) & x>0\\
%	-x & x\leq 0\\
%\end{array} \right.$ \\
%
%Ist $f$ in $x_0 = 0$ differenzierbar?\\
%
%$ \frac{f(x_0+d)-f(x_0)}{d} = \left \{ 
%\begin{array}{l  l}
%	\frac{\sin(x_0+d)-\sin(x_0)}{d} = (*) = \frac{\sin{x_0}(\cos(d)-1)}{d}+\frac{\sin(d)\cos(x_0)}{d}& x>0\\
%	\frac{-(x_0+d)-(-(x_0))}{d} & x\leq 0\\
%\end{array} \right.$ \\
%
%$\lim_{d \rightarrow 0^-} \frac{f(x_0+d)-f(x_0)}{d} = \left \{ 
%\begin{array}{l  l}
%	\frac{\sin(x_0+d)-\sin(x_0)}{d} = (*) = \frac{\sin{x_0}(\cos(d)-1)}{d}+\frac{\sin(d)\cos(x_0)}{d}& x>0\\
%	\frac{-(x_0+d)-(-(x_0))}{d} & x\leq 0\\
%\end{array} \right.$ \\
%
%$(*) = \frac{\sin(x_0+d)-\sin(x_0)}{d} = \frac{\sin(x_0)\cos(d)+\sin(d)\cos(x_0) - \sin(x_0)}{d} = \frac{\sin{x_0}(\cos(d)-1)+\sin(d)\cos(x_0)}{d}= \frac{\sin{x_0}(\cos(d)-1)}{d}+\frac{\sin(d)\cos(x_0)}{d}$\\\\
